\section{Conclusion}
\label{sec:conclusion}

In this paper, we evaluated the effect data sharing can have on the effectiveness of NIDS models against DDoS attacks. We suspected that, given the faults in NIDS models with SIDS having high false negative rates and AIDS’ threshold for detection being undefined, sharing extracted features from datasets of cyberattacks could help these models build an “immune system” to future attacks they may have otherwise missed. Our two main datasets, CIC-IDS and Aposemat IoT-23, were both synthetically generated for security reasons, and we processed the data to have features indicative of whether an attack occurred or not. Through out machine learning model and analysis, we found that sharing data of benign events is helpful for NIDS models and that sharing malicious data shows promise but needs further investigation. Our results indicated that the size of the training dataset may not be a large factor in the effectiveness of the model, and in many cases augmenting the data with foreign datasets may be beneficial to the detection system. There, of course, if further work to be done and ethical considerations to take into account, but we found that the process of sharing features of datasets including cyberattacks shows promise for the integrety of NIDS and network security. 
