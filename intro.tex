\section{Introduction}
\label{sec:intro}
\begin{intro}

The rapid advancement and expansion of the internet has resulted in 
an explosion in the number of Internet of Things (IoT) devices and a huge increase in the network size. This has resulted in a huge increase in attack surface that attackers can exploit. Cybersecurity Ventures estimates that the costs incurred due to cybercrime globally will grow by 15\% every year over the next 5 years, reaching \$10.5 trillion USD annually by 2025. 

The most promiment type of attack is the Distributed Denial of Service (DDoS) attack. In this type, a network of attackers attempts to stop legitimate users' access to a specific network resource. The word ``distributed'' refers to the nature of these attacks, as the nodes performing denial of service may be scattered across IP ranges and thus hard to block en masse. This forces the victim to inspect all traffic and differentiate legitimate traffic from anomalous traffic, an extremely challenging proposition.

In order to avoid the costly consequences of DDoS attacks, prevention is preferable to treatment. However, the effectiveness of prevention depends on how well the Intrusion Detection System (IDS) works. Network-based intrusion detection systems (NIDS) are attack detection mechanisms that detect attacks by constantly monitoring traffic for both malicious and benign behavior. However, detecting DDoS attacks is not easy, especially since novel attacks are generated frequently, rapidly making previous attack signatures outdated. One potential solution is data sharing. This allows entities that have not experienced new types of attacks to learn their signatures, effectively using other networks' experience as an immune system. This helps unexposed networks react quickly to these attacks and avoid being caught unprepared.

Data sharing can take many forms: \eg raw data, extracted features, or models. Raw data is rarely shared due to its huge privacy and security risks. As such, this study is an initial attempt to explore the feasibility of sharing extracted features to improve the effectiveness of NIDS models against DDoS attacks. Our research questions and answers are:
\begin{enumerate}
    \item \textbf{Can data sharing improve the effectiveness of NIDS models against DDoS?} Yes.
    \item \textbf{What kind of data (benign, malicious or both?) will be helpful to share?} We confirm that sharing benign data is helpful, however more research is needed to see if malicious data has an effect.
    \item \textbf{How much data should be shared?} For most types of attacks, the percentage of data shared does not have a big effect on the effectiveness on the model, however there are a few specific types for which we observe a massive increase in reliability the more data is shared.
    \item \textbf{Under what circumstances will data sharing be useful?} We show that data sharing is seemingly most useful when the sample is both large and diverse.
    
\end{enumerate}








\end{intro}
