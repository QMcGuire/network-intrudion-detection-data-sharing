\begin{abstract}
Cyberattacks on enterprise networks pose a tremendous threat to business operations today. Defending against the ever-changing landscape of threats and normal user traffic is time-consuming and labor-intensive. As a result, there is an ongoing push across many sectors to use artificial intelligence and machine learning models to automate security incident detection and response. In practice, however, there are 2 roadblocks to AI/ML-enabled workflows. (1) Any single enterprise may not have enough data to train a reliable model to detect new attack campaigns or model normal behavior. (2) An enterprise may have insufficient confidence in model outputs or hypotheses over a short period of time frame, introducing undesirable tradeoffs between false positives (\textit{i.e.} blocking legitimate users) and false negatives (\textit{i.e.} missing attacks). 

Ideally, sharing data can help address both of these problems. By sharing data (or generative models of training data), enterprises can train better discriminative models for classifying emerging attacks that may not be visible from any single vantage point. By comparing the output of these discriminative models, enterprises can implement more reliable policies (\eg\ blocked lists) that build on shared insights. Unfortunately, this information is rarely shared (if at all) due to concerns about business privacy.

Given data sharing's risk of privacy and security loss to end users and workers, it is only justified if it brings about proven increases in the anomaly detection success rate. Data sharing can take many forms: \eg\ raw data, extracted features, or models. Raw data is rarely shared due to its huge privacy and security risks; as such, this study is an initial attempt to explore the feasibility of sharing extracted features to improve Network Intrusion Detection (NIDS). A few questions we would like to explore in this study are: (1) Can data sharing improve NIDS? (2) What kind of data when shared can improve NIDS? (3) How much data should be shared? (4) Under what circumstances will data sharing improve NIDS?

\end{abstract}
